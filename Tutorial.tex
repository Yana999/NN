%задаем формат документа
\documentclass[12pt,a4paper,preview]{article}
\usepackage[english,russian]{babel}
%вставка картинок
\usepackage{graphicx}
%какие типы файлов будут восприниматься как картинки
\DeclareGraphicsExtensions{.pdf,.png,.jpg}
\graphicspath{{images/}}

\usepackage{geometry} % Меняем поля страницы
\geometry{left=2cm}% левое поле
\geometry{right=1.5cm}% правое поле
\geometry{top=1cm}% верхнее поле
\geometry{bottom=2cm}% нижнее поле

\begin{document}
%межстрочный интервал
\linespread{1.5}

\section*{\huge{Отчет по заданию Kaggle. \\Выполнила Абдраманова Яна, БПМ-17-2.}}

\hfill \break

\textbf{Набор данных:}\\
Представленный в работе набор данных показывает сведения о клиентах банка и состоит из 10 000 клиентов, указывающих свой возраст, зарплату, семейное положение, лимит кредитной карты, категорию кредитной карты и т. д. \\

\textbf{Цель работы:}\\
Визуализировать наиболее релятивные переменные, влияющие на клиентов. \\
\textbf{Основная часть:}\\
Импортируем данные и проверяем корректность работы.\\
Таблица содержит:\\
 \begin{itemize}
   \item 
   CLIENTNUM - Номер клиента. Уникальный идентификатор клиента, владеющего счетом;
   \item
   Attrition\_Flag - Активность клиента - если счет закрыт, то 1 иначе 0;
   \item 
   Customer\_Age — возраcт клиента;
   \item 
   Dependent\_count - число иждивенцев;
   \item
   Education\_Level - образовательная квалификация владельца счета (пример: выпускник средней школы, колледжа и т. д.);
   \item
   Gender — пол клиента (M = мужчина, F = женщина);
   \item
   Marital\_Status - Женат, Холост, Разведен, Неизвестен;
   \item
   Income\_Category -категория годового дохода владельца счета;
   \item
   Card\_Category - тип карты (Синяя, серебряная, золотая, Платиновая);
   \item
   Months\_on\_book - период взаимоотношений с банком.
 \end{itemize}
подключаем библиотеки:\\
\noindent\texttt{
import numpy as np\\
import pandas as pd\\
import matplotlib.pyplot as plt\\
import sys\\}

Загружаем данные, преобразовываем, заменяем лексические значения цифрами и удаляем NaN.\\
Функция correct\_indexing(dataset) корректирует индексы для дальнейшей работы.\\
Функция catagory\_churn\_unchurn(type\_of\_card, data) производит набор данных на основе типа кредитной карты клиента.\\
Функция plotter(column, Group) строит графики данного столбца 1 против столбца 2 и для 5 наборов данных.
\begin{enumerate}
\item полный набор
\item данных синие держатели кредитных карт
\item серебряные держатели кредитных карт
\item золотые держатели кредитных карт
\item платиновые держатели кредитных карт
\end{enumerate}
Графики находятся в этом конкретном порядке.
Результат работы программы:
\\

\begin{figure}[h!] 
\center{\includegraphics[scale=0.75]{Result2.png}}
\caption{Результат работы функции plotter.}
\end{figure}
\begin{figure}
\center{\includegraphics[scale=0.75]{Result3.png}}
\caption{Contacts\_Counts\_12\_mon показывает слабую отрицательную корреляцию с Attirition\_flag клиента, но очень последовательна, и эта связь может быть использована для прогнозирования будущих вспенивающихся клиентов.}
\end{figure}
\begin{figure}
\center{\includegraphics[scale=0.75]{Result5.png}}
\caption{Результат работы функции plotter.}
\end{figure}
\begin{figure}
\center{\includegraphics[scale=0.75]{Result6.png}}
\caption{Total\_Revolving\_Bal показывает сильную положительную корреляцию с Attirition\_flag клиента и очень последовательна, и эта связь может быть использована для прогнозирования будущих вспенивающихся клиентов}
\end{figure}
\begin{figure}
\center{\includegraphics[scale=0.75]{Result8.png}}
\caption{Результат работы функции plotter.}
\end{figure}
\begin{figure}
\center{\includegraphics[scale=0.75]{Result9.png}}
\caption{Total\_Amt\_Chng\_Q4\_Q1 показывает сильную положительную корреляцию с Attirition\_flag клиента и очень последовательна, и эта связь может быть использована для прогнозирования будущих вспенивающихся клиентов}
\end{figure}
\end{document}
